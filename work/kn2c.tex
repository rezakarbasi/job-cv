\resumeSubheadingReza
{Robotic Researcher}{\rezaLink{https://aras.kntu.ac.ir/kn2c/}{KN2C Robotic Lab}}
{05/2015 - 09/2017}{Tehran, Iran}
% {Contributed as an Electronic and Control Engineer in the SSL(Small Size football League) team, focusing on the design and maintenance of robot circuit boards and implementing control algorithms, including \textbf{PID} controllers for the wheel's speed and \textbf{Kalman Filter} to enhance the precision of detecting the ball in the playground.}
% {
%     \vspace{-12pt}
%     \begin{itemize}
%         % \item{Streamlined robot circuitry as an Electronics and Control Engineer with the KN2C Robotic Lab's SSL team, designing, soldering, and maintaining boards with Xmega and ST processors for enhanced motor control. Achieved greater agility and competition performance through precise PID controller calibration.}
%         \item{Enhanced robot performance in KN2C's SSL team by designing, soldering, and maintaining Xmega and ST processor boards for motor control and precise \textbf{PID} calibration.}
%         \item{Improved ball detection accuracy by implementing a \textbf{Kalman Filter} with the Armadillo library in \textbf{C++}.}
%         % \item{Crafted a real-time GUI in C\# using Visual Studio for match condition monitoring, facilitating immediate strategic decisions and enhancing team coordination.}
%     \end{itemize}
% }
{
    \vspace{-12pt}
    \begin{itemize}
        % \item{Led the electronics team in KN2C's SSL, enhancing robot performance by designing, soldering, and maintaining Xmega and ST processor boards for motor control and precise \textbf{PID} calibration. Fostered collaboration with mechanical and software teams to integrate system improvements.}
        \item{Led KN2C's SSL electronics team, improving robot performance by designing, soldering, and maintaining Xmega and ST processor boards for motor control and \textbf{PID} calibration. Collaborated with mechanical and software teams to integrate system enhancements.}
        
        \item{Boosted ball detection precision by implementing a \textbf{Kalman Filter} in \textbf{C++}, leveraging the Armadillo library.}
        \item{Crafted a real-time GUI in C\# using Visual Studio for simultaneous condition monitoring of all robots, enhancing strategic decision-making.}
    \end{itemize}
}


% {KN2C Robotic Lab is a robotic lab that works on various robotic devices and projects from UAVs to Small Size football leagues. I was there as an electronic and control engineer in the SSL team (Small Size League).}
% \resumeItemListStart
%   \resumeItem{Designer and maintainer of the main boards of our soccer players. The board had an Xmega processor and 4 ST processors to drive motors. We applied higher-level commands (wheel velocities) on each BLDC motor using a PID controller.}
%   \resumeItem{Implementing a \textbf{Kalman Filter} on the estimated position of the ball (based on the camera) to achieve a more stable strategy during gameplay. Besides, I implemented a PID controller on our robot to control the wheel's velocity.}
%   \resumeItem{Developing a GUI to monitor the robots' condition during the game, connected to the main transmitter board to receive data and display essential flags for each player.}
% \resumeItemListEnd
